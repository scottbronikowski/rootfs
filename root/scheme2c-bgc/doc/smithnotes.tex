\documentclass[10pt]{article}
\usepackage{fullpage}
\usepackage{parskip}
\usepackage{newcent}
\usepackage{s2c}

\title{\StoC\ notes for \emph{An Introduction to Scheme}}
\author{Joel F. Bartlett}
\date{}

\begin{document}

\maketitle

\emph{An Introduction to Scheme}, by Jerry D. Smith, is a recent
text on the programming language Scheme.  Rather than being
directed at a specific implementation of Scheme, it attempts
to stick close to the dialect defined in the \emph{Revised$^4$
Report on the Algorithmic Language Scheme} that is the base
for many implementations including \StoC.  This document
provides section notes to point out the differences between TI
PC Scheme used in the text and \StoC.  The user will also
find it helpful to read \emph{An Introduction to \StoC\ in 19
Prompts} and have the \emph{\StoC\ Index to the Revised$^4$
Report on the Algorithmic Language Scheme} and the
\emph{Revised$^4$ Report on the Algorithmic Language Scheme}
available for reference.

\subsection*{1.8 (((((())))))}

\StoC\ does not have an internal editor.  Instead, users
use the editor of their choice (which may or may not have
parentheses matching) and then use \textbf{load} to load the file
into the Scheme system.

\subsection*{1.10 PC Scheme and the Listener}

This is what the interaction on page 9 looks like in \StoC:

\begin{quote}
\begin{verbatim}
> (+ 3 2)
5
> (load "examples.sc")
SQUARE
"examples.sc"
> (square 2)
4
> (exit)
\end{verbatim}
\end{quote}

User input is prompted by ``\texttt{>}'', \StoC\ files end with the
suffix ``.sc'', and the command primitives \textbf{\%c} and \textbf{\%d} are
not supported.  User input is not evaluated until the user
types return.

\subsection*{2.2 The Scheme Listener}

\StoC\ is started by the command \textbf{sci}.  Once the command
is executed, the window looks like this:

\begin{quote}
\begin{verbatim}
csh >sci
Scheme->C -- 01sep91jfb -- Copyright 1989 Digital Equipment Corporation
>
\end{verbatim}
\end{quote}

When the evaluation of an expression results in an error, the
debugger is entered.  It prints an error message followed by a
procedure call traceback.  It then prompts the user with ``\texttt{>>}''
to allow commands to inspect the state of the computation
where the error occurred.  For now, simply type control-D to
return to the main read-eval-print loop.
\begin{quote}
\begin{verbatim}
> (square 2)
***** SQUARE Top-level symbol is undefined
(SQUARE 2) in ENV-0
(EVAL ...)
(SCREP_REP ...)
(READ-EVAL-PRINT ...)
>> ^D
> (load "examples.sc")
SQUARE
"examples.sc"
> (square 2)
4
> (exit)
\end{verbatim}
\end{quote}

\subsection*{2.3 Simple Arithmetic}

Most \StoC\ systems do not have bignums.  Numbers are
represented as either 29-bit integers or 64-bit floating point
values.

Exercise for the student:  Add bignums to \StoC.

The boolean constant for true is \textbf{\#t} and false is \textbf{\#f}.
In keeping with tradition, both the empty list \textbf{()} and \textbf{\#f}
are considered to be false.  It is good programming practice
to not use the empty list \textbf{()} as a synonym for \textbf{\#f}, and in
IEEE compliant Scheme's your program won't work as the empty
list \textbf{()} is a synonym for \textbf{\#t}!

\subsection*{2.4.2 Constants}

The constants \textbf{\#!true}, \textbf{\#!false}, and \textbf{\#!null} are not implemented,
use \textbf{\#t}, \textbf{\#f}, and \textbf{'()} respectively. The character
constants \textbf{\#\textbackslash{}backspace}, \textbf{\#\textbackslash{}page}, and \textbf{\#\textbackslash{}rubout} are not
implemented.

\subsection*{2.4.4 Literal Expressions}

The value of \textbf{\#f} is false, represented as \textbf{\#f}, which is
not the same as the empty list \textbf{()}.  While both \textbf{\#f} and
\textbf{()} are considered to be false when evaluating a boolean
expression, they are not equivalent.  Note that the empty list
is not a self-evaluating constant.  In order to avoid an
error, one must quote it when entering it into Scheme:

\begin{quote}
\begin{verbatim}
> ()
> (
***** EVAL Argument contains an item that is not self-evaluating: ()
(EVAL ...)
(SCREP_REP ...)
(READ-EVAL-PRINT ...)
>> ^D
> '()
()
>
\end{verbatim}
\end{quote}

\subsection*{3.2 The Global Environment}

There is only one top level environment that contains both
user and system definitions.

\subsection*{5.2 Logical Operators}

Since \textbf{\#f} and the empty list \textbf{()} are different, \textbf{\#f} is
always returned when a predicate returns false:

\begin{quote}
\begin{verbatim}
> (<= 4 3)
#F
\end{verbatim}
\end{quote}

\subsection*{5.7 begin}

In the text, the programming convention is that one calls
\textbf{newline} and then calls \textbf{display}.  In printing to the
terminal with \StoC, it is better to call \textbf{display} and
then call \textbf{newline} as a newline is not automatically
generated when Scheme prompts the user for additional input.

\subsection*{8.2 Characters}

The following \#\textbackslash{}\emph{char-name} forms are supported: \textbf{\#\textbackslash{}formfeed},
\textbf{\#\textbackslash{}linefeed}, \textbf{\#\textbackslash{}newline}, \textbf{\#\textbackslash{}return}, \textbf{\#\textbackslash{}space}, and \textbf{\#\textbackslash{}tab}.

\subsection*{8.4.1 Character Predicates}

The user need not define these as they are part of the system.

\subsection*{8.4.3 String Conversion Functions}

See the \StoC\ documentation for information about
\textbf{number-\texttt{>}string} and \textbf{string-\texttt{>}number}.

\subsection*{9.4 User-defined Port Operations}

Ports in \StoC\ are represented as a pair of the symbol
\textbf{port} and the procedure that implements it.  For example:

\begin{quote}
\begin{verbatim}
> (current-input-port)
(PORT . #*PROCEDURE*)
>
\end{verbatim}
\end{quote}

Don't forget to put the call to \textbf{newline} after the second
call to \textbf{display} when writing \textbf{addtwo}.

\subsection*{9.6 Strings as Ports}

Strings can be opened as input ports by \textbf{open-input-string} and
as output ports by \textbf{open-output-string}.  See the index for more
details.

\subsection*{9.7 A Utility for Reading Lines: read-ln}

Users will not see either the backspace or rubout characters
as they are handled by the workstation's terminal emulator.

\subsection*{10.2 Debugging and Lexical Scope}

A pretty-print procedure is provided, \textbf{pp}, but it does not
pretty-print the text of a procedure.

Breakpoints may be set on the entry and exit of any procedure
defined in the top level environment.  Rather than adding a
call to \textbf{bkpt} as was done in the text, a user would set a
breakpoint on each call to \textbf{\texttt{<}} and observe the value of
\textbf{i}:

\begin{quote}
\begin{verbatim}
> (one-to-y-sqrd 3)

0 -calls  - (> 1 3)
0- i
1
0- ^D
0 -returns- #F
0- ^D
\end{verbatim}
\end{quote}

\begin{quote}
\begin{verbatim}
0 -calls  - (> 0 3)
0- i
0
0- ^D
0 -returns- #F
0- ^D

0 -calls  - (> -1 3)
0- i
-1
0- ^D
0 -returns- #F
0- ^D

0 -calls  - (> -2 3)
0- (top-level)
>
\end{verbatim}
\end{quote}

On each entry to \textbf{\texttt{>}}, the values of it's arguments are
printed.  The value of \textbf{i} can be examined by entering \textbf{i}
followed by a return.  When control-D or \textbf{(proceed)} is
entered, the function is evaluated and the result is printed.
To continue with the computation, enter control-D or
\textbf{(proceed)}.  Once it become clear that the program is in
error, the user is able to return to the top level
read-eval-print loop by entering \textbf{(top-level)}.

Tracing is done using the \textbf{trace} and \textbf{untrace} commands:

\begin{quote}
\begin{verbatim}
> (one-to-z-sqrd 3)
12
> (trace one-to-z-sqrd)
(ONE-TO-Z-SQRD)
> (one-to-z-sqrd 3)
(ONE-TO-Z-SQRD 3)
==> 12
12
> (untrace one-to-z-sqrd)
(ONE-TO-Z-SQRD)
>
\end{verbatim}
\end{quote}

The \textbf{bpt} command puts a breakpoint on both the entry and
exit to a procedure.  The \textbf{unbpt} command removes a
breakpoint.  Here's the example on page 141:

\begin{quote}
\begin{verbatim}
> (bpt sqr)
SQR
> (one-to-z-sqrd 3)
(ONE-TO-Z-SQRD 3)

1 -calls  - (SQR 1)
1- ^D
1 -returns- 2
1- ^D
\end{verbatim}
\end{quote}

\begin{quote}
\begin{verbatim}
1 -calls  - (SQR 2)
1- ^D
1 -returns- 4
1- ^D

1 -calls  - (SQR 3)
1- ^D
1 -returns- 6
1- (top-level)
> (unbpt sqr)
(SQR)
>
\end{verbatim}
\end{quote}

When both \textbf{sqr} and \textbf{one-to-z-sqrd} are traced, one gets
the following output.

\begin{quote}
\begin{verbatim}
> (trace one-to-z-sqrd)
(ONE-TO-Z-SQRD)
> (trace sqr)
(SQR)
> (ONE-TO-Z-SQRD 3)
(ONE-TO-Z-SQRD 3)
  (SQR 1)
  ==> 2
  (SQR 2)
  ==> 4
  (SQR 3)
  ==> 6
==> 12
12
>
\end{verbatim}
\end{quote}

Tracing the recursive function \textbf{ftl} produces the following
output.

\begin{quote}
\begin{verbatim}
> (trace ftl)
(FTL)
> (ftl 3)
(FTL 3)
  (FTL 2)
    (FTL 1)
      (FTL 0)
      ==> 1
    ==> 1
  ==> 2
==> 6
6
>
\end{verbatim}
\end{quote}

\subsection*{10.3 Debugging in a Lexically Scoped Environment}

Conditional breakpoints can be placed on procedures using the
\textbf{bpt} special form.  The second argument is a test procedure
that is either the name of a top level procedure or a lambda
expression defining a procedure.  On each entry to the
procedure, the test procedure is evaluated with the arguments
to the breakpointed procedure.  When the test procedure
returns a true value, the breakpoint is taken.

\begin{quote}
\begin{verbatim}
> (bpt one-to-n (lambda (x) (>= x 3)))
ONE-TO-N
> (mean-table 5)
=======================================
N   MEAN OF 1 TO N

1         1
2         1.5
3
0 -calls  - (ONE-TO-N 3)
0-
\end{verbatim}
\end{quote}

The first time the argument to \textbf{one-to-n} is \textbf{\texttt{>=}} 3, the
breakpoint is taken.  Once at a breakpoint, the \textbf{backtrace}
procedure allows the call stack and environments to be
inspected.

\begin{quote}
\begin{verbatim}
0- (backtrace)
(ONE-TO-N N) in ENV-0
(/ (ONE-TO-N N) N) in ENV-1
(DISPLAY (ONE-TO-N-MEAN N)) in ENV-2
(BEGIN (NEWLINE) (DISPLAY N) (DISPLAY "         ") (DIS ... in ENV-3
(EVAL ...)
(SCREP_REP ...)
(READ-EVAL-PRINT ...)
#F
0-
\end{verbatim}
\end{quote}

Environments are identified by the symbols \textbf{env-}\emph{i}.
The value of an environment is an a-list of symbols and their
values. It's often useful to use \textbf{pp} to print out an
environment.  An expression may be evaluated within a specific
environment by calling \textbf{eval} with two arguments, the
expression and the environment.

\begin{quote}
\begin{verbatim}
0- env-1
((LOCATION . "inside one-to-n-mean") (N . 3))
0- env-3
((N . 3) (\d\o\l\o\o\p . #*PROCEDURE*) (PRINT-HEADER .#*PROCEDU
RE*) (HEADER-LINE . "=======================================")(
HIGH-BOUND . 5))
0- (pp env-3)
((N . 3)
     (\d\o\l\o\o\p . #*PROCEDURE*)
     (PRINT-HEADER . #*PROCEDURE*)
     (HEADER-LINE . "=======================================")
     (HIGH-BOUND . 5))#T
0- (eval 'high-bound env-3)
5
0- ^D
0 -returns- 6
0- ^D
2
4
0 -calls  - (ONE-TO-N 4)
0- ^D
0 -returns- 10
0- ^D
2.5
5
0 -calls  - (ONE-TO-N 5)
0- ^D
0 -returns- 15
0- ^D
3
=======================================
#F
>
\end{verbatim}
\end{quote}

The same techniques that one uses to explore a program that
has hit a breakpoint can be used to investigate a running
program.  Here a simple loop is run.  When the user enters
control-C the running program is interrupted.  A breakpoint is
put on \textbf{eq?}\ and then the program is continued by typing
control-D.  Once the breakpoint is hit, \textbf{(eq?\ 0 0)} is
executed, and then \textbf{proceed} is used to change the result
returned by \textbf{eq?}, which causes the loop to complete.

\begin{quote}
\begin{verbatim}
> (let loop ((i 0)) (if (eq? i 0) (loop i) 'done))
^C
***** INTERRUPT *****
(EQ? I 0) in ENV-0
(IF (EQ? I 0) (LOOP I) 'DONE) in ENV-1
(EVAL ...)
(SCREP_REP ...)
(READ-EVAL-PRINT ...)
>> (bpt eq?)
EQ?
>> ^D

0 -calls  - (EQ? 0 0)
0- ^D
0 -returns- #T
0- (proceed #f)
DONE
>
\end{verbatim}
\end{quote}

Finally, these techniques can be used to investigate the
environment when an error occurs.  Note that when control-D is
entered to continue, Scheme returns to the top level
read-eval-print loop.

\begin{quote}
\begin{verbatim}
> (let ((i 0)) (car (car (car i))))
***** CAR Argument not a PAIR: 0
(SCRT1_$_CAR-ERROR ...)
(CAR ...)
(CAR I) in ENV-0
(CAR (CAR I)) in ENV-1
(CAR (CAR (CAR I))) in ENV-2
(EVAL ...)
(SCREP_REP ...)
(READ-EVAL-PRINT ...)
>> i
0
>> env-0
((I . 0))
>> ^D
>
\end{verbatim}
\end{quote}

Exercise for the student:  implement \textbf{assert}.

\subsection*{11.3 dir: A Utility for Listing Filenames (Implementation-specific)}

In order to implement this in \StoC\ you'll need to implement your
own version of \textbf{sort!}\ and use \textbf{(open-input-port \texttt{"}ls\texttt{"})} to generate a
list of file names.

\subsection*{11.4 format: A Utility for Formatted Output}

\StoC\ contains a procedure \textbf{format}.  See the documentation for
details.

\subsection*{13.2 Memory Organization}

Exercise for the student:  implement \textbf{append!}.

\subsection*{15.4 Macros [OPTIONAL]}

\StoC\ macros implements ``expansion passing'' macros based upon the
ideas found in \emph{Expansion-Passing Style: Beyond Conventional Macros},
1986 ACM Conference on Lisp and Functional Programming, 143--150.

The simplest form of a macro is a constant.  The arguments to the
special form \textbf{define-constant} are the symbol identifying the
constant and the expression to evaluate to calculate it's value.

\begin{quote}
\begin{verbatim}
> (define-constant radius 23)
RADIUS
> (define-constant pi 3.14159)
PI
> (define-constant circumference (* pi radius 2))
CIRCUMFERENCE
> (define-constant area (* 3.14159 (* radius radius)))
AREA
> area
1661.90111
>
\end{verbatim}
\end{quote}

The second type of macro defines an in-line procedure.  The form
\textbf{define-in-line} associates a symbol with a procedure definition.
All calls to the procedure are replaced by the lambda expression
defining the procedure.

\begin{quote}
\begin{verbatim}
> (define-in-line (plus3 x) (+ x 3))
PLUS3
> (plus3 5)
8
>
\end{verbatim}
\end{quote}

The most general form of macro expansion allows the user to
examine a procedure call and then selectively cause further
macro expansion.  The definition for \textbf{plus3} can also be written
as:

\begin{quote}
\begin{verbatim}
> (define-macro plus3
	     (lambda (form expander)
		     (expander `(+ ,(cadr form) 3) expander)))
PLUS3
> (plus3 8)
11
>				
\end{verbatim}
\end{quote}

The macro is defined by a procedure that takes two arguments:
the form to be expanded, and a procedure to do further
expansion.  It's typical action is to build an expanded form
and then call the further expansion procedure with the new form and the
further expansion procedure as arguments.  For examples of use
of this type of macro expander, the reader is directed to the
file \textbf{scrt/predef.sc} that defines the macros
used by the compiler.

N.B.  Macro definitions may not be placed inside procedure
definitions.
\end{document}
