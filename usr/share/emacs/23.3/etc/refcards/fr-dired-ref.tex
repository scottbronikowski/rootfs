% Reference Card for Dired

% Copyright (C) 2000, 2001, 2002, 2003, 2004, 2005, 2006, 2007, 2008,
%   2009, 2010, 2011  Free Software Foundation, Inc.

% Author: Evgeny Roubinchtein <eroubinc@u.washington.edu>
% French translation: Eric Jacoboni

% This file is part of GNU Emacs.

% GNU Emacs is free software: you can redistribute it and/or modify
% it under the terms of the GNU General Public License as published by
% the Free Software Foundation, either version 3 of the License, or
% (at your option) any later version.

% GNU Emacs is distributed in the hope that it will be useful,
% but WITHOUT ANY WARRANTY; without even the implied warranty of
% MERCHANTABILITY or FITNESS FOR A PARTICULAR PURPOSE.  See the
% GNU General Public License for more details.

% You should have received a copy of the GNU General Public License
% along with GNU Emacs.  If not, see <http://www.gnu.org/licenses/>.


% See dired-ref.tex.

%**start of header

% This file can be printed with 1, 2, or 3 columns per page.
% Specify how many you want here.
\newcount\columnsperpage
\columnsperpage=2

% PDF output layout.  0 for A4, 1 for letter (US), a `l' is added for
% a landscape layout.
\input pdflayout.sty
\pdflayout=(0)

\def\versionemacs{23}           % version of Emacs this is for
\def\year{2011}                 % latest copyright year

% Nothing else needs to be changed.

\def\shortcopyrightnotice{\vskip 1ex plus 2 fill
  \centerline{\small \copyright\ \year\ Free Software Foundation, Inc.
  Permissions au dos.}}

\def\copyrightnotice{
\vskip 1ex plus 2 fill\begingroup\small
\centerline{Copyright \copyright\ \year\ Free Software Foundation, Inc.}
\centerline{Pour GNU Emacs version  \versionemacs}
\centerline{Conception de Stephen Gildea}
\centerline{Mis \`a jour pour Dired en Mai 2000 par Evgeny Roubinchtein}
\centerline{Traduction fran\c{c}aise d'\'Eric Jacoboni}

Vous pouvez faire et distribuer des copies de cette carte, pourvu que
la notice de copyright et cette note de permission soient pr\'eserv\'ees
sur toutes les copies.

Pour obtenir des copies du manuel de GNU Emacs:

{\tt http://www.gnu.org/software/emacs/\#Manuals}
\endgroup}

% make \bye not \outer so that the \def\bye in the \else clause below
% can be scanned without complaint.
\def\bye{\par\vfill\supereject\end}

\newdimen\intercolumnskip       %horizontal space between columns
\newbox\columna                 %boxes to hold columns already built
\newbox\columnb

\def\ncolumns{\the\columnsperpage}

\message{[\ncolumns\space
  column\if 1\ncolumns\else s\fi\space per page]}

\def\scaledmag#1{ scaled \magstep #1}

% This multi-way format was designed by Stephen Gildea October 1986.
% Note that the 1-column format is fontfamily-independent.
\if 1\ncolumns                  %one-column format uses normal size
  \hsize 4in
  \vsize 10in
  \voffset -.7in
  \font\titlefont=\fontname\tenbf \scaledmag3
  \font\headingfont=\fontname\tenbf \scaledmag2
  \font\smallfont=\fontname\sevenrm
  \font\smallsy=\fontname\sevensy

  \footline{\hss\folio}
  \def\makefootline{\baselineskip10pt\hsize6.5in\line{\the\footline}}
\else                           %2 or 3 columns uses prereduced size
  \hsize 3.4in
  \vsize 10in
  \hoffset -.75in
  \voffset -.745in
  \font\titlefont=cmbx10 \scaledmag2
  \font\headingfont=cmbx10 \scaledmag1
  \font\smallfont=cmr6
  \font\smallsy=cmsy6
  \font\eightrm=cmr8
  \font\eightbf=cmbx8
  \font\eightit=cmti8
  \font\eighttt=cmtt8
  \font\eightmi=cmmi8
  \font\eightsy=cmsy8
  \textfont0=\eightrm
  \textfont1=\eightmi
  \textfont2=\eightsy
  \def\rm{\eightrm}
  \def\bf{\eightbf}
  \def\it{\eightit}
  \def\tt{\eighttt}
  \normalbaselineskip=.8\normalbaselineskip
  \normallineskip=.8\normallineskip
  \normallineskiplimit=.8\normallineskiplimit
  \normalbaselines\rm           %make definitions take effect

  \if 2\ncolumns
    \let\maxcolumn=b
    \footline{\hss\rm\folio\hss}
    \def\makefootline{\vskip 2in \hsize=6.86in\line{\the\footline}}
  \else \if 3\ncolumns
    \let\maxcolumn=c
    \nopagenumbers
  \else
    \errhelp{You must set \columnsperpage equal to 1, 2, or 3.}
    \errmessage{Illegal number of columns per page}
  \fi\fi

  \intercolumnskip=.46in
  \def\abc{a}
  \output={%                    %see The TeXbook page 257
      % This next line is useful when designing the layout.
      %\immediate\write16{Column \folio\abc\space starts with \firstmark}
      \if \maxcolumn\abc \multicolumnformat \global\def\abc{a}
      \else\if a\abc
        \global\setbox\columna\columnbox \global\def\abc{b}
        %% in case we never use \columnb (two-column mode)
        \global\setbox\columnb\hbox to -\intercolumnskip{}
      \else
        \global\setbox\columnb\columnbox \global\def\abc{c}\fi\fi}
  \def\multicolumnformat{\shipout\vbox{\makeheadline
      \hbox{\box\columna\hskip\intercolumnskip
        \box\columnb\hskip\intercolumnskip\columnbox}
      \makefootline}\advancepageno}
  \def\columnbox{\leftline{\pagebody}}

  \def\bye{\par\vfill\supereject
    \if a\abc \else\null\vfill\eject\fi
    \if a\abc \else\null\vfill\eject\fi
    \end}
\fi

% we won't be using math mode much, so redefine some of the characters
% we might want to talk about
\catcode`\^=12
\catcode`\_=12

\chardef\\=`\\
\chardef\{=`\{
\chardef\}=`\}

\hyphenation{mini-buf-fer}
\hyphenation{de-le-tion}

\parindent 0pt
\parskip 1ex plus .5ex minus .5ex

\def\small{\smallfont\textfont2=\smallsy\baselineskip=.8\baselineskip}

% newcolumn - force a new column.  Use sparingly, probably only for
% the first column of a page, which should have a title anyway.
\outer\def\newcolumn{\vfill\eject}

% title - page title.  Argument is title text.
\outer\def\title#1{{\titlefont\centerline{#1}}\vskip 1ex plus .5ex}

% section - new major section.  Argument is section name.
\outer\def\section#1{\par\filbreak
  \vskip 3ex plus 2ex minus 2ex {\headingfont #1}\mark{#1}%
  \vskip 2ex plus 1ex minus 1.5ex}

\newdimen\keyindent

% beginindentedkeys...endindentedkeys - key definitions will be
% indented, but running text, typically used as headings to group
% definitions, will not.
\def\beginindentedkeys{\keyindent=1em}
\def\endindentedkeys{\keyindent=0em}
\endindentedkeys

% paralign - begin paragraph containing an alignment.
% If an \halign is entered while in vertical mode, a parskip is never
% inserted.  Using \paralign instead of \halign solves this problem.
\def\paralign{\vskip\parskip\halign}

% \<...> - surrounds a variable name in a code example
\def\<#1>{{\it #1\/}}

% kbd - argument is characters typed literally.  Like the Texinfo command.
\def\kbd#1{{\tt#1}\null}        %\null so not an abbrev even if period follows

% beginexample...endexample - surrounds literal text, such a code example.
% typeset in a typewriter font with line breaks preserved
\def\beginexample{\par\leavevmode\begingroup
  \obeylines\obeyspaces\parskip0pt\tt}
{\obeyspaces\global\let =\ }
\def\endexample{\endgroup}

% key - definition of a key.
% \key{description of key}{key-name}
% prints the description left-justified, and the key-name in a \kbd
% form near the right margin.
% First hfill tweaked from 0.75 to 0.8 to allow for longer descriptions.
\def\key#1#2{\leavevmode\hbox to \hsize{\vtop
  {\hsize=.8\hsize\rightskip=1em
  \hskip\keyindent\relax#1}\kbd{#2}\hfil}}

\newbox\metaxbox
\setbox\metaxbox\hbox{\kbd{M-x }}
\newdimen\metaxwidth
\metaxwidth=\wd\metaxbox

% metax - definition of a M-x command.
% \metax{description of command}{M-x command-name}
% Tries to justify the beginning of the command name at the same place
% as \key starts the key name.  (The "M-x " sticks out to the left.)
\def\metax#1#2{\leavevmode\hbox to \hsize{\hbox to .75\hsize
  {\hskip\keyindent\relax#1\hfil}%
  \hskip -\metaxwidth minus 1fil
  \kbd{#2}\hfil}}

% threecol - like "key" but with two key names.
% for example, one for doing the action backward, and one for forward.
\def\threecol#1#2#3{\hskip\keyindent\relax#1\hfil&\kbd{#2}\hfil\quad
  &\kbd{#3}\hfil\quad\cr}

% I cannot figure out how to make all dired-x
% commands fit on a page in two-column format
\def\dx{\bf (DX)}

% Set to non-zero to check for layout problems.
\overfullrule 0pt
\nopagenumbers

%**end of header


\title{Carte de r\'ef\'erence de Dired}

\centerline{(bas\'e sur Dired de GNU Emacs \versionemacs)}
\centerline{Les commandes marqu\'ees par \dx{} n\'ecessitent dired-x}

% trim this down to fit everything on one page
% \section{G\'en\'eral}
% Avec dired, vous pouvez \'editer la liste des fichiers d'un r\'epertoire
% (et, \'eventuellement, ses r\'epertoires au format 'ls -lR').

% L'\'edition d'un r\'epertoire signifie que vous pouvez visiter,
% renommer, copier, compresser, compiler des fichiers. Dans le tampon
% d'\'edition, vous pouvez modifier les attributs des fichiers, leur
% appliquer des commandes
% shell ou ins\'erer des sous-r\'epertoires. Vous pouvez � marquer � des
% fichiers pour qu'ils soient supprim\'es plus tard ou pour leur
% appliquer des commandes ; cela peut \^etre fait pour un seul fichier \`a
% la fois ou pour un ensemble de fichiers correspondant \`a certains
% crit\`eres (fichiers correspondant \`a une expression rationnelle
% donn\'ee, par exemple).

% On se d\'eplace dans le tampon \`a l'aide des commandes habituelles de
% d\'eplacement du curseur. Les lettres ne s'ins\`erent plus mais servent
% \`a ex\'ecuter des commandes, les chiffres (0-9) sont des param\`etres pr\'efixes.

% La plupart des commandes agissent soit sur tous les fichiers marqu\'es,
% soit sur le fichier courant s'il n'y a pas de fichier marqu\'e. On
% utilise un param\`etre pr\'efixe pour agir sur les NUM fichiers suivants
% (ou les NUM pr\'ec\'edents si NUM $<$ 0). Le param\`etre pr\'efixe '1' sert
% \`a op\'erer sur le fichier courant uniquement. Les param\`etres pr\'efixes
% ont priorit\'e sur les marques. Les commandes lan\c{c}ant un
% sous-processus sur un groupe de fichiers afficheront une liste des
% fichiers pour lesquels le sous-processus a \'echou\'e. Taper y tentera
% de vous expliquer ce qui a pos\'e probl\`eme.

% Lorsque l'on \'edite plusieurs fichiers dans un unique tampon, chaque
% r\'epertoire agit comme une page : C-x [ et C-x ] peuvent donc servir
% \`a se d\'eplacer dans ces r\'epertoires.

\section{Lancer et sortir de Dired}

\key{lancer dired}{C-x d} \key{\'edite le r\'epertoire du fichier que l'on
  est en train d'\'editer}{C-x C-j\dx} \key{quitter dired}{q}

\section{Commandes de d\'eplacement}

\key{ligne pr\'ec\'edente}{p}
\key{ligne suivante}{n}
\key{ligne de r\'epertoire pr\'ec\'edente}{<}
\key{ligne de r\'epertoire suivante}{>}
\key{fichier marqu\'e suivant}{M-\}}
\key{fichier marqu\'e pr\'ec\'edent}{M-\{}
\key{sous-r\'epertoire pr\'ec\'edent}{M-C-p}
\key{sous-r\'epertoire suivant}{M-C-n}
\key{r\'epertoire p\`ere}{^}
\key{premier sous-r\'epertoire}{M-C-d}

\section{Commandes avec la souris}
\metax{visiter le fichier}{Mouse_Button_2}
\metax{ouvrir un menu}{Control-Mouse_Button_3}

\section{Actions immediates sur les fichiers}

\key{visiter le fichier courant}{f}
\key{visualiser le fichier courant}{v}
\key{visiter le fichier courant dans une autre fen\^etre}{o}
%%\key{visiter le fichier courant dans un autre cadre}{w}
%%\key{afficher le fichier courant}{C-u o}
\key{cr\'eer un nouveau sous-r\'epertoire}{+}
\key{comparer le fichier sous le point avec celui sous la marque}{=}

\section{Marquer et \^oter les marques des fichiers}

\key{marquer un fichier ou un sous-r\'epertoire pour de futures commandes}{m}
\key{\^oter la marque d'un fichier ou de tous les fichiers d'un sous-r\'epertoire}{u}
\key{\^oter la marque de tous les fichiers d'un tampon}{M-delete}
\key{marquer les fichiers ayant une extension donn\'ee}{* .}
\key{marquer tous les sous-r\'epertoires}{* /}
\key{marquer tous les liens symboliques}{* @}
\key{marquer tous les ex\'ecutables}{* *}
\key{inverser le marquage}{* t}
\key{marquer tous les fichiers du sous-r\'epertoire courant}{* s}
\key{marquer les fichiers dont les noms correspondent \`a une expression
  rationnelle}{* \%}
\key{modifier les marques par un caract\`ere diff\'erent}{* c}
\key{marquer les fichiers pour lesquels une expression Elisp renvoie t}{* (\dx}

\section{Modifier le tampon Dired}

\key{ins\'erer un sous-r\'epertoire dans ce tampon}{i}
\key{supprimer les fichiers marqu\'es de la liste}{k}
\key{supprimer le listing d'un sous-r\'epertoire}{C-u k}
\key{relire tous les r\'epertoires (conserve toutes les marques)}{g}
\key{bascule le tri sur le nom/date du r\'epertoire courant}{s}
\key{\'edite les options de ls }{C-u s}
\key{r\'ecup\`ere les marques, les lignes cach\'ees, etc.}{C-_}
\key{cache tous les sous-r\'epertoires}{M-\$}
\key{cache ou fait appara\^\i{}tre le sous-r\'epertoire}{\$}

\section{Commandes sur les fichiers marqu\'es ou sp\'ecifi\'es par le pr\'efixe}

\key{copier le(s) fichier(s)}{C}
\key{renommer un fichier ou d\'eplacer des fichiers dans un autre r\'epertoire}{R}
\key{changer le propri\'etaire d'un (des) fichier(s)}{O}
\key{changer le groupe d'un (des) fichier(s)}{G}
\key{changer le mode d'un (des) fichier(s)}{M}
\key{imprimer le(s) fichier(s)}{P}
\key{convertir le(s) nom(s) de fichier(s) en minuscules}{\% l}
\key{convertir le(s) nom(s) de fichier(s) en majuscules}{\% u}
\key{supprimer les fichiers marqu\'es (pas ceux ayant un `flag')}{D}
%%\key{uuencoder ou uudecoder le(s) fichier(s)}{U}
\key{compresser ou d\'ecompacter le(s) fichier(s)}{Z}
\key{lancer info sur le fichier}{I\dx}
\key{cr\'eer un (des) lien(s) symbolique(s)}{S}
\key{cr\'eer des liens symboliques relatifs}{Y}
\key{cr\'eer un (des) liens physique(s)}{H}
\key{rechercher une expression rationnelle dans des fichiers}{A}
\key{remplacer interactivement une expression rationnelle}{Q}
\key{byte-compiler des fichiers}{B}
\key{charger le(s) fichier(s)}{L}
\key{lancer une commande shell sur le(s) fichier(s)}{!}

\section{Mettre un flag sur les fichiers \`a d\'etruire}
%% Hack for overfull line. Proper fix?
\leftline{\bf Les commandes qui \^otent les marquent suppriment les}
\leftline{\bf flags de suppression}
\key{placer un flag de suppression sur le fichier}{d}
%%\key{sauvegarder et supprimer le flag de suppression}{delete}
\key{placer un flag sur tous les fichiers de sauvegarde (dont les noms
  se terminent par \~{})}{\~{}}
\key{placer un flag sur tous les fichiers de sauvegarde automatique}{\#}
\key{placer un flag sur les diff\'erents fichiers interm\'ediaires}{\% \&}
\key{placer un flag sur les sauvegardes num\'erot\'ees (finissant par
  .\~{}1\~{}, .\~{}2\~{}, etc.)}{.}
\key{ex\'ecuter les suppressions demand\'ees (fichiers ayant un flag)}{x}
\key{placer un flag sur les fichiers correspondant \`a une expression
  rationnelle }{\% d}

\section{Commandes sur les expressions rationnelles}

\key{marquer les noms de fichiers correspondant \`a une expression
  rationnelle}{\% m}
\key{copier les fichiers marqu\'es par une expression rationnelle}{\% C}
\key{renommer les fichiers marqu\'es par une expression rationnelle}{\% R}
\key{lien physique}{\% H}
\key{lien symbolique}{\% S}
\key{lien symbolique avec chemin relatif}{\% Y}
\key{marquer pour une suppression}{\% d}

\section{Dired et Find}
%% Hack for overfull line (should be \metax). Proper fix?
\key{fichier(s) dired dont le nom correspond \`a un motif}{M-x
  find-names-dired}
\metax{fichier(s) dired contenant un motif}{M-x find-grep-dired}
\metax{fichier(s) dired bas\'es sur ce que produit \kbd{find}}{M-x find-dired}

\section{Obtenir de l'aide}

\key{aide sur dired}{h}
\key{r\'esum\'e de dired (aide succinte) et trace d'erreur}{?}

\copyrightnotice

\bye

